\documentclass{article}
\usepackage{amsmath}
\usepackage{siunitx}
\title{Algoritmo de Bellman - Ford}
\author{Humberto Alcocer}
\date{3 de Mayo, 2019}

\begin{document}
\maketitle

\section{Introducción}

El algoritmo de \textit{Bellman - Ford} nos permite calcular el camino más
corto en un grafo ponderado. Este algoritmo es menos eficiente que el algoritmo de
\textit{Dijkstra}, sin embargo es usado dado que éste soporta pesos con valores negativos.

\section{Definición de Algoritmo}

\begin{itemize}
  \item Sea $S$ = nodo origen.
  \item Sea $dij$ = costo asociado a la trayectoria de nodos $iej$
  \item
    \begin{math}
      dij= \left\{ \begin{array}{lcc}
        0 & \textnormal{cuando} & i = j \\
        \\ > 0 & \textnormal{cuando} & i \neq j \textnormal{ y existe un enlace directo entre } iej \\
        \\ \infty & \textnormal{cuando} & i \neq j \textnormal{ y \textbf{no} existe un enlace directo entre } iej
        \end{array}
      \right.
    \end{math}
  \item $D_n^{(h+1)}$ = Costo en curso, obtenido por el algoritmo para la trayectoria entre los nodos $s$ y $n$.
  \item $D_n^{(h+1)} = \min [D_j^{(h)} + djn]$
\end{itemize}

\end{document}