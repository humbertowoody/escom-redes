\documentclass{article}
\usepackage{fullpage}
\usepackage{amsmath}
\usepackage{siunitx}
\title{Capacidad de Canal}
\author{Humberto Alcocer}
\date{22 de Marzo, 2019}

\begin{document}
\maketitle

\section{Introducción}

La \textit{capacidad de canal} se refiere a lo que comúnmente confundimos con \textit{ancho de banda}.
El hecho de que lo confundamos de una forma tan común es debido a que durante muchísimo tiempo el ancho de banda estaba relacionado directamente con la capacidad de canal; principalmente durante la época de uso de \textit{Token Ring} que empleaba cable de tipo Coaxial. Posteriormente se ha redefinido debido a que la frecuencia de operación del medio físico no se encuentra directamente relacionada, al menos en la actualidad, con la capacidad de canal.

\section{Definición}

La \textit{capacidad de canal}, es la máxima velocidad a la cual los datos pueden ser transmitidos sobre un canal de comunicaciones con cierta fidelidad. Se mide en \textit{bits por segundo} (\textit{bps}) y existen 2 criterios para medirla:

\begin{itemize}
  \item \textit{Nyquist}: $C = 2B\log_{2}(v)$ donde $v$ es el \textbf{número de señales}.
  \item \textit{Shannon}: $C = B\log_{2}(1+\frac{s}{n})$
\end{itemize}

Cada uno de los criterios es dependiente de los datos proporcionados por los problemas en cuestión y servirán para desarrollar los ejercicios a modo.

\section{Ejemplos}

A continuación se describen una serie de ejmplos sobre la \textbf{capacidad de canal}, para cada uno se describen los pasos a seguir así como los despejes necesarios.

\subsection{Ejemplo 1}

Para operar a 9600 bps se usa un sistema de señalización digital.

\begin{enumerate}
  \item Si cada elemento de señal codifica una palabra de 4 bits, ¿Cuál es el ancho de banda mínimio necesario?
  \begin{subequations}
    \begin{equation}
      9600 \si{bps} = 2B(4)
    \end{equation}
    \begin{equation}
      9600 \si{bps} = 8B
    \end{equation}
    \begin{equation}
      B = \frac{9600}{8} = 1200 \si{\hertz} = 1.2 \si{\kilo\hertz}
    \end{equation}
    \item ¿Y para palabras de 8 bits?
    \begin{equation}
      B = \frac{9600}{16} = 600 \si{\hertz}
    \end{equation}
  \end{subequations}
\end{enumerate}

\subsection{Ejemplo 2}

¿Cuál es la capacidad para un canal que opera en el rango de frecuencias entre 400\si{\hertz}  y 700\si{\hertz}  con una relación $\frac{S}{N}$ de 3dB?

\begin{subequations}
  \begin{equation}
    B = 700\si{\hertz} - 400\si{\hertz} = 300\si{\hertz}
  \end{equation}
  \begin{equation}
    3dB = 10\log_{10}(\frac{S}{N})
  \end{equation}
  \begin{equation}
    \frac{3}{10}=10^{\frac{3}{10}}= 1.99
  \end{equation}
  \begin{equation}
    C = B \log_{2}(1+\frac{S}{N})
  \end{equation}
  \begin{equation}
    C = (300)\log_{2}(1+1.99)
  \end{equation}
  \begin{equation}
    C = 427.04\si{bps}
  \end{equation}
\end{subequations}

\subsection{Ejemplo 3}

Se desea construir un fax que sea capaz de transmitir una hoja tamaño carte con una resolución de 300\si{dpi} en blanco y negro empleando una línea telefónica con un ancho de banda de 4\si{\kilo\hertz} y una relación señal-ruido de 24\si{dB}.

\begin{enumerate}
  \begin{subequations}
    \item ¿Es posible realizar la transmisión en menos de 1 minuto?

    Considerando el área de la hoja $a = (8.5)(11) = 93.5 \si{pulgadas^{2}}$.

    Teniendo $300ppp$ ¿Cuántos puntos tengo en $1\si{pulgada^{2}}$? La respuesta será calcular: $(300\si{ppp})(300\si{ppp}) = 90,000 \text{ puntos por}\si{pulgada^{2}}$.

    \begin{equation}
      \frac{90,000\text{ puntos}}{1\si{pulgada^{2}}}
      =
      \frac{x}{93.5\si{pulgada^{2}}}
    \end{equation}
    \begin{equation}
      x = (8415000\si{puntos})(\log(2))
    \end{equation}
    \begin{equation}
      x = 8415000\si{bit}
    \end{equation}
    \begin{equation}
      24dB = 10\log_{10}(\frac{S}{N})
    \end{equation}
    \begin{equation}
      2.4 = \log_{10}(\frac{S}{N})
    \end{equation}
    \begin{equation}
      \frac{S}{N} = 10^{2.4} = 251.18
    \end{equation}
    \begin{equation}
      C = (4\times10^{3})\log_{2}(1+251.18) = 31913.24\si{bps} = 31.9\si{kbps}
    \end{equation}
    \begin{equation}
      \frac{31.9 \times 10^{3}\si{kbps}}{1\si{\sec}} = \frac{8415000\si{bits}}{t}
    \end{equation}
    \begin{equation}
      t = 263.68\si{\sec} = 4.39\si{\min}
    \end{equation}
    Con el resultado de la ecuación 3i podemos decir que no es posible transmitir en menos de un minuto.

    \item ¿Cuál es el tiempo mínimo para transmitir?

    $4.39\si{\min}$

    \item ¿Cuántas señales se necesitan para transmitir lo más rápido posible?
    \begin{equation}
      31.91\si{kbps}\times10^{3} = 2(4\times10^{3})\log_{2}(V)
    \end{equation}
    \begin{equation}
      31.91\si{kbps}\times10^{3} = 8\times10^{3}\log_{2}(V)
    \end{equation}
    \begin{equation}
      \frac{31.91\si{kbps}\times10^{3}}{8\times10^{3}} = \log_{2}(V)
    \end{equation}
    \begin{equation}
      \frac{31.91\si{kbps}}{8} = \log_{2}(V)
    \end{equation}
    \begin{equation}
      V = 2^{\frac{31.91}{8}} = 2^{3.99} = 15.89 \approx 16
    \end{equation}
    Respuesta: 16 señales, pero la respuesta correcta es 8 dada la capacidad de canal.

    \item Regresando a la respuesta del inciso 2:
    \begin{equation}
      C = 2(4\times10^{3})\log_{2}16
    \end{equation}
    \begin{equation}
      C = 8 \times 10^{3} (4)
    \end{equation}
    \begin{equation}
      C = 32 \times 10^{3}\si{bps} = 32\si{kbps}
    \end{equation}

    Pero 32 kbps se pasa de los 31.9 kbps que habíamos obtenido de en el insiso 1; entonces reducimos el número de señales de 16 a 8, esto con el $\log_{2}$.

    Dado el cambio en el número de señales, se tiene que recalcular el insiso 1:

    \begin{equation}
      C = 2(4\times10^{3})\log_{2}(8)
    \end{equation}
    \begin{equation}
      C = 8\times10^{3}(3)
    \end{equation}
    \begin{equation}
      C = 24\times10^{3}\si{bps}
    \end{equation}
    \begin{equation}
      \frac{24\times10^{3}\si{bps}}{1\si{\sec}}
      =
      \frac{8415000\si{bits}}{t}
    \end{equation}
    \begin{equation}
      t = 5.84\si{\min}
    \end{equation}
  \end{subequations}

\end{enumerate}

\end{document}