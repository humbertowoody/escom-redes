\documentclass{article}
\usepackage{fullpage}
\usepackage{amsmath}
\usepackage{siunitx}
\title{Capacidad de Canal}
\author{Humberto Alcocer}
\date{22 de Marzo, 2019}

\begin{document}
\maketitle

\section{Introducción}

La \textit{capacidad de canal} se refiere a lo que comúnmente confundimos con \textit{ancho de banda}.
El hecho de que lo confundamos de una forma tan común es debido a que durante muchísimo tiempo el ancho de banda estaba relacionado directamente con la capacidad de canal; principalmente durante la época de uso de \textit{Token Ring} que empleaba cable de tipo Coaxial. Posteriormente se ha redefinido debido a que la frecuencia de operación del medio físico no se encuentra directamente relacionada, al menos en la actualidad, con la capacidad de canal.

\section{Definición}

La \textit{capacidad de canal}, es la máxima velocidad a la cual los datos pueden ser transmitidos sobre un canal de comunicaciones con cierta fidelidad. Se mide en \textit{bits por segundo} (\textit{bps}) y existen 2 criterios para medirla:

\begin{itemize}
  \item \textit{Nyquist}: $C = 2B\log_{2}(v)$ donde $v$ es el \textbf{número de señales}.
  \item \textit{Shannon}: $C = B\log_{2}(1+\frac{s}{n})$
\end{itemize}

Cada uno de los criterios es dependiente de los datos proporcionados por los problemas en cuestión y servirán para desarrollar los ejercicios a modo.

\section{Ejemplos}

A continuación se describen una serie de ejmplos sobre la \textbf{capacidad de canal}, para cada uno se describen los pasos a seguir así como los despejes necesarios.

\subsection{Ejemplo 1}

Para operar a 9600 bps se usa un sistema de señalización digital.

\begin{enumerate}
  \item Si cada elemento de señal codifica una palabra de 4 bits, ¿Cuál es el ancho de banda mínimio necesario?
  \begin{subequations}
    \begin{equation}
      9600 \si{bps} = 2B(4)
    \end{equation}
    \begin{equation}
      9600 \si{bps} = 8B
    \end{equation}
    \begin{equation}
      B = \frac{9600}{8} = 1200 \si{\hertz} = 1.2 \si{\kilo\hertz}
    \end{equation}
    \item ¿Y para palabras de 8 bits?
    \begin{equation}
      B = \frac{9600}{16} = 600 \si{\hertz}
    \end{equation}
  \end{subequations}
\end{enumerate}

\subsection{Ejemplo 2}

¿Cuál es la capacidad para un canal que opera en el rango de frecuencias entre 400\si{\hertz}  y 700\si{\hertz}  con una relación $\frac{S}{N}$ de 3dB?

\begin{subequations}
  \begin{equation}
    B = 700\si{\hertz} - 400\si{\hertz} = 300\si{\hertz}
  \end{equation}
  \begin{equation}
    3dB = 10\log_{10}(\frac{S}{N})
  \end{equation}
  \begin{equation}
    \frac{3}{10}=10^{\frac{3}{10}}= 1.99
  \end{equation}
  \begin{equation}
    C = B \log_{2}(1+\frac{S}{N})
  \end{equation}
  \begin{equation}
    C = (300)\log_{2}(1+1.99)
  \end{equation}
  \begin{equation}
    C = 427.04\si{bps}
  \end{equation}
\end{subequations}

\subsection{Ejemplo 3}

Se desea construir un fax que sea capaz de transmitir una hoja tamaño carte con una resolución de 300\si{dpi} en blanco y negro empleando una línea telefónica con un ancho de banda de 4\si{\kilo\hertz} y una relación señal-ruido de 24\si{dB}.

\begin{enumerate}
  \item ¿Es posible realizar la transmisión en menos de 1 minuto? Considerando el área de la hoja $a = (8.5)(11) = 93.5 \si{pulgadas^{2}}$. Teniendo $300ppp$ ¿Cuántos puntos tengo en $1\si{pulgada^{2}}$?
\end{enumerate}

\end{document}